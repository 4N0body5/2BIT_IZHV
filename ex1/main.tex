%%%%%%%%%%%%%%%%%%%%%%%%%%%%%%%%%%%%%%%%%%%%%%%%%%%%%%%%%%%%%%%%%%%%%%%%%%%%%%%%
% Author : [Name] [Surname], Tomas Polasek (template)
% Description : First exercise in the Introduction to Game Development course.
%   It deals with an analysis of a selected title from the point of its genre, 
%   style, and mechanics.
%%%%%%%%%%%%%%%%%%%%%%%%%%%%%%%%%%%%%%%%%%%%%%%%%%%%%%%%%%%%%%%%%%%%%%%%%%%%%%%%

\documentclass[a4paper,10pt,english]{article}

\usepackage[left=2cm,right=2cm,top=1.0cm,bottom=1.5cm]{geometry}
\usepackage[utf8]{inputenc}
\usepackage{enumitem}
\newcommand{\ph}[1]{\textit{[#1]}}

\title{%
\vspace*{-1cm}
\huge Analysis of Mechanics%
\vspace*{-0.3cm}
}
\author{%
Natália Bubáková (xbubak01)%
}
\date{}

\begin{document}

\maketitle
\thispagestyle{empty}

{%
\large
\vspace{-0.4cm}
\begin{itemize}[leftmargin = 0cm]

\item[] \makebox[4.5cm]{\textbf{Title:}\hfill} \textsc{Spore}

\item[] \makebox[4.5cm]{\textbf{Released:}\hfill} \textsc{2008}

\item[] \makebox[4.5cm]{\textbf{Author:}\hfill} \textsc{Maxis, EA}

\item[] \makebox[4.5 cm]{\textbf{Primary Genre:}\hfill} \textsc{god game}

\item[] \makebox[4.5 cm]{\textbf{Secondary Genres:}\hfill} \textsc{life simulation, action RPG, real-time strategy,\\\hspace*{4.6cm}adventure, casual, \dots}

\item[] \makebox[4.5 cm]{\textbf{Style:}\hfill} \textsc{cartoon}

\end{itemize}

}
\vspace*{-0.2cm}
\section*{\centering Analysis}
\vspace*{-0.1cm}
\subsection*{Introduction}

I have selected a title Spore developed by Maxis and published by Electronic Arts in 2008, as I consider the concept that Will Wright designed, grand and innovative. Although the game itself was rather shallow than imposing. This lost potential might have been caused especially by difficulty to execute such an extensive idea that covers and intervenes numerous genres. Thus it is still perfect example to demonstrate features and mechanics within multiple genres.

\subsection*{Spore and its clasification}

Spore is a life simulation game, starting single cell, then going up through the evolution. Player takes the god role and supervises development of single species until it becomes intelligent and achieves or even outgrows certain degree of sapience.

At first glance, it is clear why game's creators decided to classify Spore primarily as a god game. Nevertheless, it does not fall neatly to any one genre as it is made up of several stages of gameplay within creature evolution.

Moving through these stages we can watch a multitude of aspects and approaches of various genres. Along the life simulation, beginning with Pac-Man-like action in the cell stage, developing to more action RPG in the creature stage and finally growing up to real-time strategy or even adventure in later stages such as tribal, civilization or space stage. This is also supplemented by the possibility to customize the creature in editor that offers some aspects of casual.

\subsection*{Stages and their aspects within genres}
Once the game is launched, player gets god-like duties to choose a planet, name it and name the species to be born there. Further they choose herbivore or carnivore diet by preference, and decide on looks of the species, names of the particular generations and others throughout the game.

Focusing on the genres interaction, the primary genre might be understood as some kind of envelope keeping all the secondary genres tight together. While these offer fundamentally different view of the game in each stage.

First stage presents a cellular phase within 2D plane with action-like game approach. The main cell object moves around, collects items, and  eats upon its diet. It can attach smaller objects for food and should avoid the bigger ones not to get eaten. It is introduced to his origin species and it pairs to evolve and get more abilities. 

The evolution of the generation is represented by customizing the character in the editor. The only limitation there are the items (body parts) that have been collected in fights with other species. Yet the results may widely vary based on player's creativity and diversity of the possibilities.

Moving to the higher stages, it gets more dynamics within 3D environment. Our role-play protagonist gets more functions moving around, that is later enhanced with group strategy abilities, building and other ways to cherish the homeland until it gets to the final phase that outgrows the planet. This is set in the universe, where gameplay gets more adventure-like character as it becomes more about reading text and making decisions.

\subsection*{Style of the game and conclusion}
Game is definitely cartoon-styled. The idea is set in some kind of fairyland and all the shapes of the creature are limited only by player's fantasy. This is even highlighted with cinematic demonstrations that accompany whole story of the species' life. Opening with the life's origin and proceeding through all the major milestones for full gameplay experience. This style is wisely chosen and neatly supports this god-like image. Nonetheless it still could be better designed or enriched with more immersive details of environment visuals.


\end{document}
